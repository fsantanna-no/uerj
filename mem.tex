\documentclass[12pt,a4paper]{article}
\usepackage[nohead,a4paper,lmargin=1.5cm,
rmargin=2.0cm,tmargin=2.0cm,bmargin=2.0cm]{geometry}

\usepackage[brazilian]{babel}
\usepackage[utf8]{inputenc}
\usepackage[T1]{fontenc}

\usepackage[pdftex]{graphicx}
\usepackage{verbatim}
\usepackage{xspace}
\usepackage{color}
\usepackage{url}

\newcommand{\code}[1] {{\small{\texttt{#1}}}}
\newcommand{\TODO}[1]{\colorbox{red}{TODO: #1}\linebreak}
\newcommand{\CEU}{\textsc{C\'{e}u}\xspace}
\newcommand{\GVT}{\emph{LuaGravity}\xspace}

\title{Memorial Descritivo}
\author{Francisco Sant'Anna}
\date{Rio de Janeiro, \today}

\begin{document}

\maketitle

\section*{Introdução}

Brasileiro, nascido no Rio de Janeiro em 1979, Doutor em Ciência da Computação 
pela PUC--Rio em 2014.
%
O meu interesse por computação veio durante a juventude, em meados da década de 
90, quando ganhei um PC com modem, no auge do período das BBSs.

Minha vida acadêmica teve início no CAp--UERJ, onde estudei desde o primário 
até o vestibular.
%
Em 1997, ingressei na turma de Engenharia da Computação da UFRJ e cursei os 
dois primeiros anos, me transferindo em 1999 para a PUC--Rio, onde me formei em 
2003.
%
Em paralelo à universidade trabalhei em diversas empresas, nas áreas de 
segurança de redes, desenvolvimento Web, e sistemas embarcados.
%
Após terminar a faculdade, atuei por três anos na área de jogos e aplicativos 
para celulares.

Em 2007 retornei à faculdade para o mestrado em linguagens de programação com o 
professor Roberto Ierusalimschy.
%
Meu interesse por linguagens de programação iniciou-se na graduação, após fazer 
algumas disciplinas específicas com o professor Roberto, e culminou com o 
projeto final que foi uma linguagem para simular circuitos digitais.
%
No mestrado desenvolvi uma extensão para Lua funcionar de forma reativa, na 
qual aplicações são guiadas por estímulos externos.
%
Além da pesquisa da dissertação, também atuei no desenvolvimento do padrão de 
TV Digital brasileira, mais especificamente na integração entre as linguagens 
NCL e Lua.
%
O doutorado continuou na linha de sistemas reativos, onde projetei a linguagem 
\CEU com foco em verificações estáticas para sistemas concorrentes de tempo 
real.
Durante esse período, fiz doutorado sanduíche na Suécia onde aplicamos \CEU à 
área de redes de sensores sem fio.
%
Após o doutorado, iniciei o pós doutorado na PUC--Rio em 2013, onde continuo 
com o desenvolvimento de \CEU.

Fui professor do Instituto de Tecnologia ORT (2\textsuperscript{o} grau e 
escola técnica) por dois anos, tendo lecionado as disciplinas de Introdução à 
Programação (1\textsuperscript{o} ano), Tópicos Especiais (Jogos e Redes de 
Sensores sem Fio, 3\textsuperscript{o} ano) e Projeto Final 
(3\textsuperscript{o} ano).
%
Na PUC--Rio, planejei em parceria com a Professora Noemi Rodriguez a disciplina 
Sistemas Reativos, que está sendo lecionada por mim pela segunda vez.

\section*{Pesquisa}

O foco principal da minha pesquisa é em linguagens concorrentes, mais 
especificamente em linguagens para sistemas reativos.
%
Em sistemas reativos, o controle de aplicações é predominantemente guiado por 
interações contínuas e em tempo real com o mundo externo.
%
Interações com o ambiente têm natureza reativa: um estímulo externo gera uma 
reação no programa que pode gerar um estímulo de volta ao ambiente.
%
Por exemplo, em um sistema de janelas, o movimento do mouse altera 
instantanemente a posição do ponteiro na tela, enquanto que o clique sobre o 
ícone \code{[X]} fecha a janela de aplicação.
%
Os principais desafios dentro desse contexto são o de manter o sistema 
``responsivo'' (i.e., com resposta idealmente imediata) e o de tratar estímulos 
simultâneos/concorrentes de forma correta.

A área de sistemas reativos me acompanhou (e me desafiou) durante a trajetória 
como programador e pesquisador:
no estágio com sistemas embarcados, no projeto final com sistemas digitais, 
como profissional na indústria de jogos, na pesquisa com TV Digital, e 
finalmente, como objeto fim na pesquisa em linguagens reativas durante o 
mestrado e doutorado.

Como pesquisador, vivenciei três projetos bem delimitados em escopo e 
experiências: o padrão de TV Digital brasileiro, a linguagem \GVT, desenvolvida 
no mestrado e a linguagem \CEU, desenvolvida no doutorado.

Entre 2007 e 2009, fui pesquisador do Laboratório Telemídia, coordenado pelo 
professor Luiz Fernando Gomes Soares, idealizador do Sistema Brasileiro de TV 
Digital (conhecido como Ginga).
%
Através do Ginga, emissoras podem distribuir aplicações que executam em 
sincronia com programas de TV, permitindo que telespectadores interajam em 
tempo real.
%
A linguagem declarativa NCL, desenvolvida para esse fim, pode ser estendida com 
scripts escritos em Lua (de maneira análoga a como JavaScript estende HTML).
%
Durante o período no Telemídia, fui o responsável pelo design, documentação e 
implementação da ponte \emph{NCLua} entre as linguagens NCL e Lua.
%
NCLua é usado para auxiliar NCL em tarefas difíceis de realizar 
declarativamente, tais como comunicação por TCP/IP e interações mais complexas 
(e.g., jogos).
%
O principal artigo referente a esse período foi publicado no \emph{WebMedia} em 
2008~\cite{nclua.webmedia}, simpósio nacional onde muitos trabalhos 
relacionados ao Ginga são publicados.
Além do artigo, fui co-autor do capítulo sobre \emph{NCLua} do livro de 
NCL~\cite{nclua.book} (baseado no conteúdo de um mini-curso dado no 
\emph{WebMedia} de 2009~\cite{nclua.shortcourse}).
Também escrevi a parte da norma ABNT referente ao 
\emph{NCLua}~\cite{nclua.abnt}.
Finalmente, escrevemos um artigo com viés mais teórico para a conferência 
\emph{DocEng} em 2009~\cite{nclua.doceng}.

O mesmo período entre 2007 e 2009 constituiu o mestrado sob orientação do 
professor Roberto.
%
O mestrado foi uma experiência transformativa de ``assimilação'' do modo de 
operação da academia.
%
A abrangência e o rigor da literatura acadêmica foram fundamentais para a 
formulação e reflexão sobre os problemas encontrados em sistemas reativos, 
assim como para o levantamento do estado da arte da área.
%Aliado pela a experiência prática com o desenvolvimento de jogos e TV Digital
%
Com o decorrer da pesquisa, identificamos duas abordagens complementares que 
visavam facilitar o desenvolvimento e verificação de sistemas reativos:
o estilo imperativo da linguagem Esterel e
o estilo declarativo das linguagens functionais reativas (FRP).
%
A dissertação de mestrado discutiu essas duas abordagens com base no design da 
linguagem \GVT, uma extensão em tempo de execução para a linguagem Lua.
%
Com o \GVT é possível escrever comandos em paralelo tais como \code{await E} 
para bloquear uma linha de execução até que o evento \code{E} ocorra (estilo 
imperativo), ou ainda expressões como \code{a=b+c} em que a mudança em \code{b} 
ou \code{c} atualiza o valor de \code{a} automaticamente (estilo FRP).
%
O trabalho com o \GVT foi publicado no \emph{SBLP} de 
2009~\cite{luagravity.sblp} e apresentado no \emph{Lua workshop} de 2009.

O período entre 2009 e os dias de hoje compreende o doutorado e o 
desenvolvimento da linguagem \CEU.
%
Resultado do mestrado, \GVT teve um caráter mais exploratório, não tão 
preocupado com um design de linguagem completo e robusto.
%
\GVT é uma extensão carregada em tempo de execução e, portanto,
atrelada à semântica de Lua, por exemplo, o uso de coleta de lixo é obrigatório 
e não existe acesso ao código fonte original ou à árvore sintática (para 
verificações estáticas).
%
Dessa maneira, vi a necessidade de recomeçar, ``dar um passo atrás, para dar 
dois a frente''.
%
\CEU nasceu com requisitos mais rigorosos: eficiência no uso de recursos, 
garantias estáticas para concorrência e apoio em uma semântica precisa.
%
Como consequência, também direcionei os casos de uso a aplicações onde
eficiência e garantias estáticas tornam-se diferencais, mesmo com alguma 
limitação em expressividade:
Redes de sensores sem fio (RSSF) executam em hardware bastante restrito, 
dependem de bateria para funcionar e são instalados em locais de acesso físico 
difícil (florestas, vulcões, etc.).
%
Utilizamos \CEU para desenvolver drivers e protocolos já existentes para RSSFs, 
com recursos equivalentes aos desenvolvidos originalmente em C.
No entanto, as implementações em \CEU têm garantias teóricas de responsividade 
e de ausência de condições de corrida em variáveis compartilhadas.
%
O design de \CEU com os casos de uso foi publicado em 2013 no 
\emph{SenSys}~\cite{ceu.sensys13}, a conferência de maior prestígio na área de 
RSSFs.
%
Com o fim do doutorado, o foco foi em expandir o poder de expressividade de 
\CEU, em particular no suporte a aplicações com alocação dinâmica.
Como resultado, a nova abstração de \emph{organismos} unifica objetos e linhas 
de execução em uma mesma construção, sem abrir mão das garantias estáticas do 
design original de \CEU.
%
Esse conceito foi apresentado no \emph{Future Programming Workshop} e 
\emph{Workshop on Reactive and Event-based Languages \& Systems} que 
aconteceram dentro do \emph{OOPSLA/SPLASH} em 2014.
O artigo completo será publicado na conferência \emph{Modularity/AOSD} em 
2015~\cite{ceu.mod15}.

\CEU é um projeto de longo prazo, com diversas opções de continuidade em 
pesquisa, ensino e também para o mercado de programadores (``indústria'').
%
Esse ano, fui convidado a apresentar \CEU no encontro anual do \emph{Working 
Group on Language Design}, grupo de trabalho vinculado ao \emph{IFIP 
(International Federation for Information Processing)}.
%
No ano passado apresentei \CEU no \emph{StrangeLoop}, conferência não acadêmica 
voltada para programadores profissionais.
%
No que diz respeito a ensino, \CEU faz parte do programa de algum curso ou 
disciplina desde 2011:
lecionei RSSF para o 3\textsuperscript{o} ano técnico da escola ORT por dois 
anos;
na PUC--Rio, planejei a disciplina eletiva \emph{Sistemas Reativos}, que esse 
ano
estará sendo dada pela segunda vez.
%
Fui informado (por e-mail, de maneira informal) sobre o uso de \CEU na 
disciplina de RSSF de 
Stanford\footnote{\url{http://web.stanford.edu/class/cs240e/}} e de 
Compiladores em 
Berkeley\footnote{\url{https://www.youtube.com/watch?v=hC5i-Wr4FuA}}.
%
\CEU particiou do \emph{Google Summer of Code} em 2014 com um estudante da 
Índia em parceria com o LabLua (onde trabalho atualmente).
%
Também estamos estabelecendo uma comunidade de 
\CEU\footnote{\url{http://www.ceu-lang.org/}}, com website,
documentação, wiki, vídeos, lista de discussão, e implementação open 
source\footnote{\url{http://github.com/fsantanna/ceu/}} para Arduino, redes de 
sensores, smarthphones e desktops.

\newpage
\section*{Títulos e Trabalhos (para análise da banca)}

\subsection*{Qualificação Acadêmica}

\begin{itemize}
\item
Graduação em Engenharia de Computação, 2003, PUC--Rio.
--- [Anexo--XXX (diploma)]
\item
Mestrado em Informática, 2009, PUC--Rio.
--- [Anexo--XXX (diploma)]
\item
Doutorado em Informática, 2013, PUC--Rio.
--- [Anexo--XXX (diploma)]
\item
Pós-Doutorado em Informática, 2013--2018 (2\textsuperscript{o} ano em 
andamento), PUC--Rio.
--- [Anexo--XXX (comprovação do departamento)]
\end{itemize}

\subsection*{Publicações e Produção Científica}

\begin{itemize}

\item Modularity 2015 (AOSD até 2011, Qualis-A1, \textbf{a ser publicado})
``Structured Synchronous Reactive Programming with Céu''~\cite{ceu.mod15}
--- [Anexo--XXX (primeira página, programa)]

\item ACM SenSys 2013 (Qualis-A1)
``Safe system-level concurrency on resource-constrained 
nodes''~\cite{ceu.sensys13}
--- [Anexo--XXX (primeira página, página na biblioteca digital da ACM)]

\item ACM DocEng 2009 (Qualis-B1)
``Relating declarative hypermedia objects and imperative objects through the 
NCL glue language''~\cite{nclua.doceng}
--- [Anexo--XXX (primeira página, página na biblioteca digital da ACM)]

\item SBLP 2009 (Qualis-B3)
``LuaGravity, a Reactive Language Based on Implicit 
Invocation''~\cite{luagravity.sblp}
--- [Anexo--XXX (primeira página, publicações sobre a linguagem Lua no site 
oficial)]

\item WebMedia 2008 (Qualis-B3)
``NCLua: objetos imperativos lua na linguagem declarativa 
NCL''~\cite{nclua.webmedia}
--- [Anexo--XXX (primeira página, página na biblioteca digital da ACM)]

\item Capítulo de livro 2009
``Programando com Objetos NCLua. Programando em NCL''~\cite{nclua.book}
--- [Anexo--XXX (capa do livro, primeira página do capítulo)]

\item Capítulo de livro 2009 (minicurso)
``Desenvolvimento de Aplicações Híbridas para TV Digital Interativa no 
Middleware Ginga''~\cite{nclua.shortcourse}
--- [Anexo--XXX (primeira página, site)]

\end{itemize}

Sobre os critérios de pontuação para produção científica (edital, Anexo II, 
Quadro 2, página 17), solicito analisar/reconsiderar se a separação entre 
publicações em periódicos e conferências acadêmicas deve ser seguida à risca,
uma vez que o documento de área da Capes para Ciência da Computação%
\footnote{\url{
http://www.capes.gov.br/images/stories/download/avaliacaotrienal/Docs_de_area/
},
Ciência da Computação, Seção IV, páginas 15--16.
} considera os dois meios igualmente importantes:
\begin{quote}
``Durante os três últimos períodos de Avaliação Trienal (2004, 2007 e 2010) a 
área de Ciência da Computação trabalhou com publicações em conferências e 
periódicos.
Na área, as publicações submetidas a conferências tradicionais passam por um 
rigoroso processo de avaliação por pares e os artigos publicados, disponíveis 
em bases de dados internacionais, são hoje tão importantes para o avanço da 
área como os melhores artigos em veículos classificados de periódicos.
Qualquer pesquisador da área de Ciência da Computação sabe que há conferências 
de enorme prestígio e que os artigos publicados nos anais dessas conferências 
são levados em alta conta em avaliações de pesquisa.
Há documentos, inclusive do IEEE, enfatizando a importância das conferências 
para a área.
Como o conjunto de publicações em conferências e periódicos é essencial para a 
avaliação da área em qualquer parte do mundo, a avaliação da produção 
bibliográfica compreende os veículos chamados tradicionalmente de periódicos e 
de anais de conferências.''
(--- [Anexo--XXX)]
\end{quote}

O critério do edital pontua periódicos e congressoes de maneira desproporcional 
(máximo de 60 e 10 pontos, respectivamente) e desconsidera por completo a 
qualidade dos congressos.
Por exemplo, artigos de 12 páginas em congressos Qualis A têm a
mesma pontuação que artigos de 5 páginas em congressos sem avaliação Qualis.

Como exemplos dentro da minha produção, a conferência SenSys tem média 
histórica de 17\% de aceitação%
\footnote{\url{
http://dl.acm.org/citation.cfm?id=2517360&preflayout=flat#source
}}, enquanto que o Modularity passa por duas fases de revisão%
\footnote{\url{http://www.aosd.net/2015/rrtrack}}:
\begin{quote}
``Modularity'15 is deeply committed to publishing works of the highest caliber.  
To this aim, two separate paper submission deadlines and review stages are 
offered. A paper accepted in any round will be published in the proceedings and 
presented at the conference. Promising papers submitted in the first round that 
are not accepted may be invited to be revised and resubmitted for review by the 
same reviewers in the second round. Authors of such invited resubmissions are 
asked to also submit a letter explaining the revisions made to the paper to 
address the reviewers' concerns. While there is no guarantee that an invited 
resubmission will be accepted, this procedure (similar to major revisions 
requested by journals) is designed to help authors of promising work get their 
papers into the conference.''
\end{quote}

\subsection*{Docência e Atividade Profissional}

\begin{itemize}
\item
Professor do Instituto de Tecnologia ORT, 2010--2012 (2 anos)
--- [Anexo--XXX (carteira de trabalho)]
\item
Professor da disciplina \emph{Sistemas Reativos} na PUC--Rio, 2014--2015 (2 
semestres)
--- [Anexo--XXX (e-mail para os alunos da graduação)]
\item
Programador de jogos para celulares, Wiz Technologies, 2004--2007 (2 anos)
--- [Anexo--XXX (carteira de trabalho)
\end{itemize}

\newpage
\bibliographystyle{acm}
\bibliography{my}

\end{document}
